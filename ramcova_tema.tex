% Metódy inžinierskej práce

\documentclass[10pt,twoside,slovak,coursepaper]{article}

\usepackage[slovak]{babel}
%\usepackage[T1]{fontenc}
\usepackage[IL2]{fontenc} % lepšia sadzba písmena Ľ než v T1
\usepackage[utf8]{inputenc}
\usepackage{graphicx}
\usepackage{url} % príkaz \url na formátovanie URL
\usepackage{hyperref} % odkazy v texte budú aktívne (pri niektorých triedach dokumentov spôsobuje posun textu)

\usepackage{cite}
%\usepackage{times}

\pagestyle{headings}

\title{Názov\thanks{Semestrálny projekt v predmete Metódy inžinierskej práce, ak. rok 2024/25, vedenie: 	doc. Ing. Ján Lang, PhD.}} % meno a priezvisko vyučujúceho na cvičeniach

\author{Matúš Majer\\[2pt]
	{\small Slovenská technická univerzita v Bratislave}\\
	{\small Fakulta informatiky a informačných technológií}\\
	{\small \texttt{xmajerm@stuba.sk}}
	}

\date{\small 30. september 2024} % upravte



\begin{document}

\maketitle

\begin{abstract}
\ldots
\end{abstract}



\section{Úvod}

Motivujte čitateľa a vysvetlite, o čom píšete. Úvod sa väčšinou nedelí na časti.

Uveďte explicitne štruktúru článku. Tu je nejaký príklad.
Základný problém, ktorý bol naznačený v úvode, je podrobnejšie vysvetlený v časti.
Dôležité súvislosti sú uvedené v častiach.
Záverečné poznámky prináša časť.

\end{document}